\section{Internal Architecture}
The lot controller consists of four internal components, namely \textit{MainComponent}, \textit{IOComponent}, \textit{EmptyComponent} and \textit{ProjectComponent}.

\textit{MainComponent} receives information about the lots in the tray and the need for calibration.
This information is distributed to the relevant components.
Furthermore it makes sure that once all lots are processed that the system will go to its final state and finishes.

\textit{IOComponent} is responsible for keeping robot $r_1$ busy.
It places wafers onto chuck $p_\mathit{in}$ and pre-measures them whenever possible.
Also whenever a wafer is at $p_\mathit{out}$ \textit{IOComponent} will place this wafer in the tray.
Information about the number of wafers to take from and to place into the tray is communicated with the \textit{MainComponent}.
\textit{ProjectComponent} and \textit{IOComponent} keep each other up-to-date about the status of $p_\mathit{in}$ and $p_\mathit{out}$.

\textit{ProjectComponent} is responsible for the measurement and projection of the wafers.
Wafers that are pre-measured and located a $p_{in}$ are moved by the \textit{ProjectComponent}, measured, projected and placed onto chuck $p_\mathit{out}$.
\textit{ProjectComponent} is also responsible for calibration whenever this is required.
\textit{ProjectComponent} receives information about the recipe and about the need for calibration from \textit{MainComponent}.
It interacts with \textit{IOComponent} about the states of $p_\mathit{in}$ and $p_\mathit{out}$.
Furthermore it utilizes the \textit{EmptyComponent} to place or remove closing wafers.

\textit{EmptyComponent} subordinates to \textit{ProjectComponent} and places or removes a closing wafer on/from $p_\mathit{meas}$ whenever desired by \textit{ProjectComponent}.

\begin{figure}
    \includegraphics[width=\textwidth]{img/internal_architecture.png}
    \caption{Internal architecture}
    \label{fig:internal_arch}
\end{figure}
Figure~\ref{fig:internal_arch} shows the different internal components as well as the interactions they perform with external systems. 

\subsection{Internal Interactions}
The four internal components will run as parallel processes.
These processes need a way to communicate with each other.
When communication takes place, one process will send information to another process which receives that information.
These send and receive actions need to take place simultaneously.
Such a multi-action can also be described as a single communication action.
The text below describes the communication actions between two components.
Each first mentioned component will send the information, whereas the second mentioned component will receive the sent information.
In our mCRL2~model (see Appendix~\ref{sec:model}) a send action is prefixed by an `s' and a receive action is prefixed by an `r'.
The allowed actions are all external interactions and communication actions.
All communication actions listed below cannot be observed by the outside world and thus can be hidden using the internal $\tau$-action.

\textbf{MainComponent to IOComponent:}
\begin{itemize}
    \item $\mathit{AvailableWafers}(n)$\\
    Communicates the amount of wafers that are on the current lot where $n \in \mathbb{N}$ is the amount of wafers.
\end{itemize}

\textbf{IOComponent to MainComponent:}
\begin{itemize}
    \item $\mathit{CompletedLot}$\\
    Indicates that all the wafers on the current lot are placed into the tray.
\end{itemize}

\textbf{MainComponent to ProjectComponent:}
\begin{itemize}
    \item $\mathit{ChangeConfig}(r, c)$\\
    Communicate a change in lot configuration which includes the recipe $r \in Q$ as well as the need for calibration $c \in \mathbb{B}$.
\end{itemize}

\textbf{ProjectComponent to MainComponent:}
\begin{itemize}
    \item $\mathit{ChangedConfig}$\\
    Indicates that the configuration has been changed.
\end{itemize}

\textbf{IOComponent to ProjectComponent:}
\begin{itemize}
    \item $\mathit{WaferReadyAtChuckIn}$\\
    Indicates that a wafer is at $p_\mathit{in}$ which has been pre-measured.
    \item $\mathit{WaferTakenAtChuckOut}$\\
    Indicates that the wafer which was on $p_\mathit{out}$ has been placed into the tray.
\end{itemize}

\textbf{ProjectComponent to IOComponent:}
\begin{itemize}
    \item $\mathit{WaferReadyAtChuckOut}$\\
    Indicates that a wafer is at $p_\mathit{out}$ which has been projected.
    \item $\mathit{WaferTakenAtChuckIn}$\\
    Indicates that the wafer which was on $p_\mathit{in}$ has been moved to $p_\mathit{meas}$.
\end{itemize}

\textbf{ProjectComponent to EmptyComponent:}
\begin{itemize}
    \item $\mathit{ProvideClosingWafer}$\\
    Request a closing wafer to be placed at $p_\mathit{meas}$.
    \item $\mathit{StoreClosingWafer}$\\
    Request the closing wafer to be removed from $p_\mathit{meas}$.
\end{itemize}

\textbf{EmptyComponent to ProjectComponent:}
\begin{itemize}
    \item $\mathit{ProvidedClosingWafer}$\\
    Indicates that a closing wafer has been placed at $p_\mathit{meas}$.
    \item $\mathit{StoredClosingWafer}$\\
    Indicates that a closing wafer has been removed from $p_\mathit{meas}$.
\end{itemize}
