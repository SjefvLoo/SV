\section{External Interactions}
Interactions describe in what manner the system behavior is controlled by the controller.
External actions specifically describe the communication between external parts of the system and the controller.
In this section the interactions of the controller, which consists of $6$ sub controllers, with the external environment will be described.
In order to properly define the external interactions, some variables will be introduced.

Let $P = \left\{p_\mathit{in}, p_\mathit{out}, p_\mathit{meas}, p_\mathit{proj}, p_\mathit{empty1}, p_\mathit{empty2}\right\}$ be the set of positions (i.e. chucks) at which wafers can be placed.
Additionally, let $\mathit{p_\mathit{tray}}$ be the location of the tray.
Furthermore, let $R = \{r_1,r_2,r_3\}$ be the set of robots.

\subsection{Observational variables}
In order to better specify the external interactions a number of observational variables are declared:
\begin{itemize}
    \item Let $c : P \mapsto \left\{\mathit{none}, \mathit{closing}, \mathit{nonclosing}, \mathit{premeasured}, \mathit{measured}, \mathit{projected}\right\}$ be a function.
        Then $c(p)$ denotes the type of wafer that is on the chuck at position $p$, where $p \in P$.
        In the initial situation both $c\left(p_\mathit{meas}\right) = \mathit{closing}$ and $c\left(p_\mathit{proj}\right) = \mathit{closing}$.
        For all other positions in the initial situation the output of $c$ is $\mathit{none}$.
    \item Let $\mathit{lock} : P \mapsto \left\{\text{False}, \text{True}\right\}$ be a function that denotes whether or not the chuck at a position is ready to be used.
        For all $p \in P$, the initial value of $\mathit{lock}(p) = \text{False}$.
\end{itemize}

\subsection{Lot Controller to Machine}
\begin{itemize}
    \item $\mathit{RMoveFromTo}(r, p_1, p_2)$ \\
        Request robot $r$ to move the wafer that is located at position $p_1$ to position $p_2$ where $r \in R$ and $p_1, p_2 \in P$.
        For these movements there is a set of legal argument combinations that are accepted, all other argument combinations are not accepted.
        Only the following argument combinations are accepted:
        \begin{align*}
            M = \{ & \left(r_1, p_\mathit{tray},   p_\mathit{in}     \right), \\
                   & \left(r_1, p_\mathit{out},    p_\mathit{tray}   \right), \\
                   & \left(r_2, p_\mathit{in},     p_\mathit{meas}   \right), \\
                   & \left(r_2, p_\mathit{meas},   p_\mathit{empty1} \right), \\
                   & \left(r_2, p_\mathit{empty1}, p_\mathit{meas}   \right), \\
                   & \left(r_3, p_\mathit{empty2}, p_\mathit{meas}   \right), \\
                   & \left(r_3, p_\mathit{meas},   p_\mathit{empty2} \right), \\
                   & \left(r_3, p_\mathit{meas},   p_\mathit{out}    \right)\} \\
        \end{align*}
        Some observational variables change for each valid $\mathit{RMoveFromTo}$:
        \begin{itemize}
            \item If $\mathit{RMoveFromTo}\left(r_1, p_\mathit{tray}, p_\mathit{in}\right)$, then $c\left(p_\mathit{in}\right) = \mathit{nonclosing}$.
                Also, $\mathit{lock}(p_\mathit{in})$ changes to $\text{True}$.
            \item If $\mathit{RMoveFromTo}\left(r_1, p_\mathit{out}, p_\mathit{tray}\right)$, then $c\left(p_\mathit{out}\right) = \mathit{none}$.
                Also, $\mathit{lock}(p_\mathit{out})$ changes to $\text{True}$.
            \item Let $r \in R$, $p_1, p_2 \in P$ and let $w = c\left(p_1\right)$ be the type of wafer currently on chuck $p_1$.
                For every valid move $\mathit{RMoveFromTo}\left(r, p_1, p_2\right)$ (i.e. $\left(r, p_1, p_2\right) \in M$), $c\left(p_2\right) = s$ and $c\left(p_1\right) = none$.
                Furthermore, both  $\mathit{lock}(p_\mathit{p_1})$ and  $\mathit{lock}(p_\mathit{p_2})$ change to $\text{True}$.
        \end{itemize}

    \item $\mathit{Calibrate}$ \\
    Request calibration to be performed.
    This changes $\mathit{lock}(p_\mathit{meas})$ and $\mathit{lock}(p_\mathit{proj})$ both to $\text{True}$.

    \item $\mathit{PreMeasureWafer}$ \\
    Request pre-measurement to be performed on the wafer at \chuckIn.
    This changes $\mathit{lock}(p_\mathit{in})$ to $\text{True}$.

    \item $\mathit{MeasureWafer}$ \\
    Request measurement to be performed on the wafer at \chuckMeas.
    This changes $\mathit{lock}(p_\mathit{meas})$ to $\text{True}$.

    \item $\mathit{ProjectWafer}$ \\
    Request an image projection of the wafer at \chuckProj.
    This changes $\mathit{lock}(p_\mathit{proj})$ to $\text{True}$.

    \item $\mathit{Swap}$ \\
    Request a swap of the chucks present on the \chuckMeas and \chuckProj.
    Some observational variables change for a $\mathit{Swap}$.
    Let $m = c\left(p_\mathit{meas}\right)$ and $p = c\left(p_\mathit{proj}\right)$ be the type of wafer currently on chuck $p_\mathit{meas}$ and $p_\mathit{proj}$ respectively.
    If $\mathit{Swap}$, then $c\left(p_\mathit{meas}\right) = p$ and $c\left(p_\mathit{proj}\right) = m$.
    Furthermore, both $\mathit{lock}(p_\mathit{meas})$ and $\mathit{lock}(p_\mathit{proj})$ change to $\text{True}$.
\end{itemize}

\subsection{Machine to Lot Controller}
\begin{itemize}
    \item $\mathit{ProvideLotInfo}(n, r, b)$ \\
    Provide the lot controller with info about the current lot in the \tray.
    When $n$ equals the amount of wafers in the lot, $r$ indicates the recipe (i.e. \recipeOne, \recipeTwo or \recipeThree) and $b$ indicates the need for calibration.
    When $b$ is $\text{True}$, the system needs to be calibrated before the lot is processed.
    When $b$ is $\text{False}$ no calibration is needed.
    This does not change any observational variables.

    \item $\mathit{RIdle(r, p_1, p_2)}$ \\
    Indicates that robot $r \in R$ has finished moving a wafer from $p_1 \in P$ to $p_2 \in P$.
    This changes $\mathit{lock}(p_1)$ to $\text{False}$ if $p_1 \in P$ and changes $\mathit{lock}(p_2)$ to $\text{False}$ if $p_2 \in P$.
    
    \item $\mathit{SwapIdle}$ \\
    Indicates that $\mathit{Swap}$ has finished.
    This changes $\mathit{lock}(p_\mathit{meas})$ and $\mathit{lock}(p_\mathit{proj})$ both to $\text{False}$.

    \item $\mathit{PreMeasured}$ \\
    Indicates that $\mathit{PreMeasureWafer}$ has finished.
    This changes $\mathit{lock}(p_\mathit{in})$ to $\text{False}$.

    \item $\mathit{Measured}$ \\
    Indicates that $\mathit{MeasureWafer}$ has finished.
    This changes $\mathit{lock}(p_\mathit{meas})$ to $\text{False}$.

    \item $\mathit{Projected}$ \\
    Indicates that $\mathit{ProjectWafer}$ has finished.
    This changes $\mathit{lock}(p_\mathit{proj})$ to $\text{False}$.

    \item $\mathit{Calibrated}$ \\
    Indicates that $\mathit{Calibrate}$ has finished.
    This changes $\mathit{lock}(p_\mathit{meas})$ and $\mathit{lock}(p_\mathit{proj})$ both to $\text{False}$.

\end{itemize}

\subsection{Lot Controller to User}
\begin{itemize}
    \item $\mathit{Finished}$ \\
    The system is finished with the current Tray, and indicates this to the operator.
    This does not change any observational variables.
\end{itemize}

\subsection{User to Lot Controller}
\todo{delete this subsection?!}
