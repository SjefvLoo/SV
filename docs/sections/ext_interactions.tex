\section{External Interactions}\label{sec:ext_interactions}
Interactions describe in what manner the system behavior is controlled by the controller.
External actions specifically describe the communication between external parts of the system and the controller.
In this section the interactions of the controller, which consists of $6$ sub controllers, with the external environment will be described.
In order to properly define the external interactions, some variables will be introduced.

Let $P = \left\{p_\mathit{in}, p_\mathit{out}, p_\mathit{meas}, p_\mathit{proj}, p_\mathit{empty1}, p_\mathit{empty2}\right\}$ be the set of positions (i.e. chucks) at which wafers can be placed.
Additionally, let $\mathit{p_\mathit{tray}}$ be the location of the tray.
Let $T$ be $P \cup \{p_\mathit{tray}\}$.
Furthermore, let $R = \{r_1,r_2,r_3\}$ be the set of robots.
Let $Q$ be the set of recipes: $Q = \{Recipe_1, Recipe_2, Recipe_3\}$.


\subsection{Lot Controller to Machine}
\begin{itemize}
    \item $\mathit{RMoveFromTo}(r, p_1, p_2)$ \\
        Request robot $r$ to move the wafer that is located at position $p_1$ to position $p_2$ where $r \in R$ and $p_1, p_2 \in P$.
        For these movements there is a set of legal argument combinations that are accepted, all other argument combinations are not accepted.
        Only the following argument combinations are accepted:
        \begin{align*}
            M = \{ & \left(r_1, p_\mathit{tray},   p_\mathit{in}     \right), \\
                   & \left(r_1, p_\mathit{out},    p_\mathit{tray}   \right), \\
                   & \left(r_2, p_\mathit{in},     p_\mathit{meas}   \right), \\
                   & \left(r_2, p_\mathit{meas},   p_\mathit{empty1} \right), \\
                   & \left(r_2, p_\mathit{empty1}, p_\mathit{meas}   \right), \\
                   & \left(r_3, p_\mathit{empty2}, p_\mathit{meas}   \right), \\
                   & \left(r_3, p_\mathit{meas},   p_\mathit{empty2} \right), \\
                   & \left(r_3, p_\mathit{meas},   p_\mathit{out}    \right)\} \\
        \end{align*}

    \item $\mathit{Calibrate}$ \\
    Request calibration to be performed.
   
    \item $\mathit{PreMeasureWafer}$ \\
    Request pre-measurement to be performed on the wafer at \chuckIn.
   
    \item $\mathit{MeasureWafer}$ \\
    Request measurement to be performed on the wafer at \chuckMeas.
 
    \item $\mathit{ProjectWafer}$ \\
    Request an image projection of the wafer at \chuckProj.
 
    \item $\mathit{Swap}$ \\
    Request a swap of the chucks present on the \chuckMeas and \chuckProj.
\end{itemize}

\subsection{Machine to Lot Controller}
\begin{itemize}
    \item $\mathit{ProvideLotInfo}(n, r, b)$ \\
    Provide the lot controller with info about the current lot in the \tray.
    When $n$ equals the amount of wafers in the lot, $r$ indicates the recipe (i.e. \recipeOne, \recipeTwo or \recipeThree) and $b$ indicates the need for calibration.
    When $b$ is $\text{True}$, the system needs to be calibrated before the lot is processed.
    When $b$ is $\text{False}$ no calibration is needed.
    This does not change any observational variables.

    \item $\mathit{RIdle(r, p_1, p_2)}$ \\
    Indicates that robot $r \in R$ has finished moving a wafer from $p_1 \in T$ to $p_2 \in T$.
    
    \item $\mathit{SwapIdle}$ \\
    Indicates that $\mathit{Swap}$ has finished.

    \item $\mathit{PreMeasured}$ \\
    Indicates that $\mathit{PreMeasureWafer}$ has finished.

    \item $\mathit{Measured}$ \\
    Indicates that $\mathit{MeasureWafer}$ has finished.

    \item $\mathit{Projected}$ \\
    Indicates that $\mathit{ProjectWafer}$ has finished.

    \item $\mathit{Calibrated}$ \\
    Indicates that $\mathit{Calibrate}$ has finished.

\end{itemize}

\subsection{Lot Controller to User}
\begin{itemize}
    \item $\mathit{Finished}$ \\
    The system is finished with the current Tray, and indicates this to the operator.
\end{itemize}