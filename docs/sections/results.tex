\section{Verification}\label{sec:results}
The lot controller requires information about the lots on the tray through \textit{ProvideLotInfo} to run.
\textit{ProvideLotInfo} takes 3 arguments: a natural number, a recipe and a boolean.
Furthermore \textit{ProvideLotInfo} represents a single lot, the only restriction to the number of lots is that it is finite.
The model can't be checked for all possibile inputs therefore the process \textit{TrayInputLoop} is declared which provides the lot controller with \text{ProvideLotInfo} commands.
\textit{TrayInputLoop} takes 2 parameters: $\mathit{maxLots} \in \mathbb{N}$ and $\mathit{maxWafersPerLot} \in \mathbb{N}$.
\textit{TrayInputLoop} calls \textit{ProvideLotInfo} with at most $\mathit{maxWafersPerLot}$ wafers at most $\mathit{maxLots}$ times.
The number of possible traces for \textit{TrayInputLoop} is:
$$(\mathit{maxLots}+1) \cdot 3 \cdot \mathit{maxWafersPerLot} \cdot 2$$
This formula represent the possible number of lots multiplied by the number of recipes multiplied by the maximum number of wafers per lot multiplied by two (i.e.\ whether to calibrate or not)). $\mathit{maxLots} = 10$ and $\mathit{maxWafersPerLot} = 24$ are used during validation of the model.

\subsection{Execution}
The model is verified on a machine with the following specifications:\\
Operating system: Ubuntu 17.04\\
Cpu: Intel(R) Pentium(R) CPU B980 @ 2.40GHz\\
RAM: 8 GB\\

The mCRL2 toolset version ``201707.1 (Release)'' was used to verify the models.

The following command is used to generate the \textit{lps} file:\\
\begin{lstlisting}[style=sh,caption={mcf2lps}] 
mcrl22lps -lregular2 "${MCRL2_NAME}.mcrl2" "${MCRL2_NAME}.lps"
\end{lstlisting}
Where \$\{MCRL2\_NAME\}.mcrl2 refers to the mcrl2 file (without filetype suffix) containing the model which can be found in appendix \ref{sec:model}.\\
The following command is used to verify an \textit{mcf} file:\\
\begin{lstlisting}[style=sh,caption={mcf check}] 
lps2pbes "${MCRL2_NAME}.lps" "${file%.mcf}.pbes" "-f${file}" \
                && pbes2bool "${file%.mcf}.pbes" --rewriter=jittyc
\end{lstlisting}
Where \$\{file\} refers to an mcf file containing the a modal formula which can be found in appendix \ref{sec:mcf}.\\
\subsection{Results}
First the \text{lps} file was generated using the \text{mcf2lps} command above. Second every \textit{mcf} file is checked using the \textit{mcf check} command.
Table \ref{tab:results1024} below shows the results.
\begin{table}[!hb]
    \centering
    \begin{tabular}{|l|l|l|}
        \hline
        \textbf{Mcf} & \textbf{Result} & \textbf{Execution Time} \\ \hline
			Auxilary 1 & true & 4m44s \\ \hline
			Requirement 1 & true & 0m43s \\ \hline
			Requirement 2a & true & 0m41s \\ \hline
			Requirement 2b & true & 1m49s \\ \hline
			Requirement 2c & true & 0m12s \\ \hline
			Requirement 3a & true & 2m48s \\ \hline
			Requirement 3b & true & 0m37s \\ \hline
			Requirement 3c & true & 0m38s \\ \hline
			Requirement 3d & true & 0m34s \\ \hline
			Requirement 3e & true & 0m33s \\ \hline
			Requirement 3f & true & 0m33s \\ \hline
			Requirement 4 & true & 4m04s \\ \hline
			Requirement 5 & true & 7m22s \\ \hline
			Requirement 6 & true & 0m42s \\ \hline
			Requirement 7 & true & 1m44s \\ \hline
			Requirement 8 & true & 1m57s \\ \hline
			Requirement 9 & true & 1m03s \\ \hline
			Requirement 10 & true & 0m48s \\ \hline
			Requirement 11 & true & 3m51s \\ \hline
			Requirement 12 & true & 0m31s \\ \hline
			Requirement 13 & true & 8m31s \\ \hline
			Requirement 14 & true & 1m53s \\ \hline
			Requirement 15 & true & 1m57s \\ \hline
			Requirement 16 & true & 0m37s \\ \hline
			Requirement 17 & true & 1m58s \\ \hline
			Requirement 18 & true & 2m17s \\ \hline
    \end{tabular}
    \caption{Results for at most ten lots and at most 24 wafers per lot}
    \label{tab:results1024}
\end{table}
